% Options for packages loaded elsewhere
\PassOptionsToPackage{unicode}{hyperref}
\PassOptionsToPackage{hyphens}{url}
%
\documentclass[
]{article}
\usepackage{lmodern}
\usepackage{amssymb,amsmath}
\usepackage{ifxetex,ifluatex}
\ifnum 0\ifxetex 1\fi\ifluatex 1\fi=0 % if pdftex
  \usepackage[T1]{fontenc}
  \usepackage[utf8]{inputenc}
  \usepackage{textcomp} % provide euro and other symbols
\else % if luatex or xetex
  \usepackage{unicode-math}
  \defaultfontfeatures{Scale=MatchLowercase}
  \defaultfontfeatures[\rmfamily]{Ligatures=TeX,Scale=1}
\fi
% Use upquote if available, for straight quotes in verbatim environments
\IfFileExists{upquote.sty}{\usepackage{upquote}}{}
\IfFileExists{microtype.sty}{% use microtype if available
  \usepackage[]{microtype}
  \UseMicrotypeSet[protrusion]{basicmath} % disable protrusion for tt fonts
}{}
\makeatletter
\@ifundefined{KOMAClassName}{% if non-KOMA class
  \IfFileExists{parskip.sty}{%
    \usepackage{parskip}
  }{% else
    \setlength{\parindent}{0pt}
    \setlength{\parskip}{6pt plus 2pt minus 1pt}}
}{% if KOMA class
  \KOMAoptions{parskip=half}}
\makeatother
\usepackage{xcolor}
\IfFileExists{xurl.sty}{\usepackage{xurl}}{} % add URL line breaks if available
\IfFileExists{bookmark.sty}{\usepackage{bookmark}}{\usepackage{hyperref}}
\hypersetup{
  pdftitle={HW4MSIA401},
  pdfauthor={Qiaozhen Wu, Qiana Yang, Dian Yu, Yijun Wu},
  hidelinks,
  pdfcreator={LaTeX via pandoc}}
\urlstyle{same} % disable monospaced font for URLs
\usepackage[margin=1in]{geometry}
\usepackage{color}
\usepackage{fancyvrb}
\newcommand{\VerbBar}{|}
\newcommand{\VERB}{\Verb[commandchars=\\\{\}]}
\DefineVerbatimEnvironment{Highlighting}{Verbatim}{commandchars=\\\{\}}
% Add ',fontsize=\small' for more characters per line
\usepackage{framed}
\definecolor{shadecolor}{RGB}{248,248,248}
\newenvironment{Shaded}{\begin{snugshade}}{\end{snugshade}}
\newcommand{\AlertTok}[1]{\textcolor[rgb]{0.94,0.16,0.16}{#1}}
\newcommand{\AnnotationTok}[1]{\textcolor[rgb]{0.56,0.35,0.01}{\textbf{\textit{#1}}}}
\newcommand{\AttributeTok}[1]{\textcolor[rgb]{0.77,0.63,0.00}{#1}}
\newcommand{\BaseNTok}[1]{\textcolor[rgb]{0.00,0.00,0.81}{#1}}
\newcommand{\BuiltInTok}[1]{#1}
\newcommand{\CharTok}[1]{\textcolor[rgb]{0.31,0.60,0.02}{#1}}
\newcommand{\CommentTok}[1]{\textcolor[rgb]{0.56,0.35,0.01}{\textit{#1}}}
\newcommand{\CommentVarTok}[1]{\textcolor[rgb]{0.56,0.35,0.01}{\textbf{\textit{#1}}}}
\newcommand{\ConstantTok}[1]{\textcolor[rgb]{0.00,0.00,0.00}{#1}}
\newcommand{\ControlFlowTok}[1]{\textcolor[rgb]{0.13,0.29,0.53}{\textbf{#1}}}
\newcommand{\DataTypeTok}[1]{\textcolor[rgb]{0.13,0.29,0.53}{#1}}
\newcommand{\DecValTok}[1]{\textcolor[rgb]{0.00,0.00,0.81}{#1}}
\newcommand{\DocumentationTok}[1]{\textcolor[rgb]{0.56,0.35,0.01}{\textbf{\textit{#1}}}}
\newcommand{\ErrorTok}[1]{\textcolor[rgb]{0.64,0.00,0.00}{\textbf{#1}}}
\newcommand{\ExtensionTok}[1]{#1}
\newcommand{\FloatTok}[1]{\textcolor[rgb]{0.00,0.00,0.81}{#1}}
\newcommand{\FunctionTok}[1]{\textcolor[rgb]{0.00,0.00,0.00}{#1}}
\newcommand{\ImportTok}[1]{#1}
\newcommand{\InformationTok}[1]{\textcolor[rgb]{0.56,0.35,0.01}{\textbf{\textit{#1}}}}
\newcommand{\KeywordTok}[1]{\textcolor[rgb]{0.13,0.29,0.53}{\textbf{#1}}}
\newcommand{\NormalTok}[1]{#1}
\newcommand{\OperatorTok}[1]{\textcolor[rgb]{0.81,0.36,0.00}{\textbf{#1}}}
\newcommand{\OtherTok}[1]{\textcolor[rgb]{0.56,0.35,0.01}{#1}}
\newcommand{\PreprocessorTok}[1]{\textcolor[rgb]{0.56,0.35,0.01}{\textit{#1}}}
\newcommand{\RegionMarkerTok}[1]{#1}
\newcommand{\SpecialCharTok}[1]{\textcolor[rgb]{0.00,0.00,0.00}{#1}}
\newcommand{\SpecialStringTok}[1]{\textcolor[rgb]{0.31,0.60,0.02}{#1}}
\newcommand{\StringTok}[1]{\textcolor[rgb]{0.31,0.60,0.02}{#1}}
\newcommand{\VariableTok}[1]{\textcolor[rgb]{0.00,0.00,0.00}{#1}}
\newcommand{\VerbatimStringTok}[1]{\textcolor[rgb]{0.31,0.60,0.02}{#1}}
\newcommand{\WarningTok}[1]{\textcolor[rgb]{0.56,0.35,0.01}{\textbf{\textit{#1}}}}
\usepackage{graphicx,grffile}
\makeatletter
\def\maxwidth{\ifdim\Gin@nat@width>\linewidth\linewidth\else\Gin@nat@width\fi}
\def\maxheight{\ifdim\Gin@nat@height>\textheight\textheight\else\Gin@nat@height\fi}
\makeatother
% Scale images if necessary, so that they will not overflow the page
% margins by default, and it is still possible to overwrite the defaults
% using explicit options in \includegraphics[width, height, ...]{}
\setkeys{Gin}{width=\maxwidth,height=\maxheight,keepaspectratio}
% Set default figure placement to htbp
\makeatletter
\def\fps@figure{htbp}
\makeatother
\setlength{\emergencystretch}{3em} % prevent overfull lines
\providecommand{\tightlist}{%
  \setlength{\itemsep}{0pt}\setlength{\parskip}{0pt}}
\setcounter{secnumdepth}{-\maxdimen} % remove section numbering

\title{HW4MSIA401}
\author{Qiaozhen Wu, Qiana Yang, Dian Yu, Yijun Wu}
\date{10/18/2020}

\begin{document}
\maketitle

\hypertarget{r-markdown}{%
\subsection{R Markdown}\label{r-markdown}}

\hypertarget{section}{%
\subsection{4.10}\label{section}}

\begin{Shaded}
\begin{Highlighting}[]
\KeywordTok{library}\NormalTok{(car)}
\end{Highlighting}
\end{Shaded}

\begin{verbatim}
## Loading required package: carData
\end{verbatim}

\begin{Shaded}
\begin{Highlighting}[]
\KeywordTok{library}\NormalTok{(dplyr)}
\end{Highlighting}
\end{Shaded}

\begin{verbatim}
## 
## Attaching package: 'dplyr'
\end{verbatim}

\begin{verbatim}
## The following object is masked from 'package:car':
## 
##     recode
\end{verbatim}

\begin{verbatim}
## The following objects are masked from 'package:stats':
## 
##     filter, lag
\end{verbatim}

\begin{verbatim}
## The following objects are masked from 'package:base':
## 
##     intersect, setdiff, setequal, union
\end{verbatim}

\begin{Shaded}
\begin{Highlighting}[]
\NormalTok{data <-}\StringTok{ }\KeywordTok{c}\NormalTok{(}\DecValTok{8}\NormalTok{,}\DecValTok{1}\NormalTok{,}\DecValTok{1}\NormalTok{,}\DecValTok{1}\NormalTok{,}
          \DecValTok{8}\NormalTok{,}\DecValTok{1}\NormalTok{,}\DecValTok{1}\NormalTok{,}\DecValTok{0}\NormalTok{,}
          \DecValTok{8}\NormalTok{,}\DecValTok{1}\NormalTok{,}\DecValTok{1}\NormalTok{,}\DecValTok{0}\NormalTok{,}
          \DecValTok{0}\NormalTok{,}\DecValTok{0}\NormalTok{,}\DecValTok{9}\NormalTok{,}\DecValTok{1}\NormalTok{,}
          \DecValTok{0}\NormalTok{,}\DecValTok{0}\NormalTok{,}\DecValTok{9}\NormalTok{,}\DecValTok{1}\NormalTok{,}
          \DecValTok{0}\NormalTok{,}\DecValTok{0}\NormalTok{,}\DecValTok{9}\NormalTok{,}\DecValTok{1}\NormalTok{,}
          \DecValTok{2}\NormalTok{,}\DecValTok{7}\NormalTok{,}\DecValTok{0}\NormalTok{,}\DecValTok{1}\NormalTok{,}
          \DecValTok{2}\NormalTok{,}\DecValTok{7}\NormalTok{,}\DecValTok{0}\NormalTok{,}\DecValTok{1}\NormalTok{,}
          \DecValTok{2}\NormalTok{,}\DecValTok{7}\NormalTok{,}\DecValTok{0}\NormalTok{,}\DecValTok{1}\NormalTok{,}
          \DecValTok{0}\NormalTok{,}\DecValTok{0}\NormalTok{,}\DecValTok{0}\NormalTok{,}\DecValTok{10}\NormalTok{,}
          \DecValTok{0}\NormalTok{,}\DecValTok{0}\NormalTok{,}\DecValTok{0}\NormalTok{,}\DecValTok{10}\NormalTok{,}
          \DecValTok{0}\NormalTok{,}\DecValTok{0}\NormalTok{,}\DecValTok{0}\NormalTok{,}\DecValTok{10}\NormalTok{)}
\NormalTok{webster <-}\StringTok{ }\KeywordTok{data.frame}\NormalTok{(}\KeywordTok{cbind}\NormalTok{(}\KeywordTok{matrix}\NormalTok{(data, }\DataTypeTok{nrow =} \DecValTok{12}\NormalTok{, }\DataTypeTok{ncol =} \DecValTok{4}\NormalTok{, }\DataTypeTok{byrow =} \OtherTok{TRUE}\NormalTok{), }\KeywordTok{rep}\NormalTok{(}\DecValTok{10}\NormalTok{, }\DecValTok{12}\NormalTok{)))}
\KeywordTok{colnames}\NormalTok{(webster) <-}\StringTok{ }\KeywordTok{c}\NormalTok{(}\StringTok{"x1"}\NormalTok{, }\StringTok{"x2"}\NormalTok{,}\StringTok{"x3"}\NormalTok{, }\StringTok{"x4"}\NormalTok{, }\StringTok{"y"}\NormalTok{)}
\end{Highlighting}
\end{Shaded}

\hypertarget{a}{%
\subsubsection{(a)}\label{a}}

\begin{Shaded}
\begin{Highlighting}[]
\KeywordTok{cor}\NormalTok{(webster[, }\DecValTok{-5}\NormalTok{])}
\end{Highlighting}
\end{Shaded}

\begin{verbatim}
##             x1          x2         x3         x4
## x1  1.00000000  0.05230658 -0.3433818 -0.4976109
## x2  0.05230658  1.00000000 -0.4315953 -0.3706964
## x3 -0.34338179 -0.43159531  1.0000000 -0.3551214
## x4 -0.49761095 -0.37069641 -0.3551214  1.0000000
\end{verbatim}

\hypertarget{no-correlation-coefficent-exceed-0.5-in-absolute-value.-therefore-correlations-do-no-indicate-multilinearity.}{%
\subsubsection{No correlation coefficent exceed 0.5 in absolute value.
Therefore, correlations do no indicate
multilinearity.}\label{no-correlation-coefficent-exceed-0.5-in-absolute-value.-therefore-correlations-do-no-indicate-multilinearity.}}

\hypertarget{b}{%
\subsubsection{(b)}\label{b}}

\begin{Shaded}
\begin{Highlighting}[]
\NormalTok{fit <-}\StringTok{ }\KeywordTok{lm}\NormalTok{(y }\OperatorTok{~}\NormalTok{., }\DataTypeTok{data =}\NormalTok{ webster)}
\KeywordTok{vif}\NormalTok{(fit)}
\end{Highlighting}
\end{Shaded}

\begin{verbatim}
##       x1       x2       x3       x4 
## 178.2874 158.0460 257.9074 289.3750
\end{verbatim}

\hypertarget{all-vif-exceed-150-and-the-greatest-is-289.375.-therefore-vifs-indicate-multilinearity.}{%
\subsubsection{All vif exceed 150 and the greatest is 289.375.
Therefore, vifs indicate
multilinearity.}\label{all-vif-exceed-150-and-the-greatest-is-289.375.-therefore-vifs-indicate-multilinearity.}}

\hypertarget{section-1}{%
\subsection{4.11}\label{section-1}}

\hypertarget{a-1}{%
\subsubsection{(a)}\label{a-1}}

\begin{Shaded}
\begin{Highlighting}[]
\NormalTok{mpg<-}\StringTok{ }\KeywordTok{read.csv}\NormalTok{(here}\OperatorTok{::}\KeywordTok{here}\NormalTok{(}\StringTok{"mpg.csv"}\NormalTok{))}
\KeywordTok{cor}\NormalTok{(mpg)}
\end{Highlighting}
\end{Shaded}

\begin{verbatim}
##                     mpg  cylinders displacement horsepower     weight
## mpg           1.0000000 -0.7776175   -0.8051269 -0.7784268 -0.8322442
## cylinders    -0.7776175  1.0000000    0.9508233  0.8429834  0.8975273
## displacement -0.8051269  0.9508233    1.0000000  0.8972570  0.9329944
## horsepower   -0.7784268  0.8429834    0.8972570  1.0000000  0.8645377
## weight       -0.8322442  0.8975273    0.9329944  0.8645377  1.0000000
## acceleration  0.4233285 -0.5046834   -0.5438005 -0.6891955 -0.4168392
##              acceleration
## mpg             0.4233285
## cylinders      -0.5046834
## displacement   -0.5438005
## horsepower     -0.6891955
## weight         -0.4168392
## acceleration    1.0000000
\end{verbatim}

\hypertarget{yes-there-is-multicollinearity-since-there-are-values-like-0.95-and-0.8-which-are-all-over-0.7.}{%
\subsubsection{Yes there is multicollinearity since there are values
like 0.95 and 0.8 which are all over
0.7.}\label{yes-there-is-multicollinearity-since-there-are-values-like-0.95-and-0.8-which-are-all-over-0.7.}}

\#\#\#(b)

\begin{Shaded}
\begin{Highlighting}[]
\NormalTok{fit_}\DecValTok{5}\NormalTok{ <-}\StringTok{ }\KeywordTok{lm}\NormalTok{(mpg }\OperatorTok{~}\StringTok{ }\NormalTok{cylinders}\OperatorTok{+}\NormalTok{displacement}\OperatorTok{+}\NormalTok{horsepower}\OperatorTok{+}\NormalTok{weight }\OperatorTok{+}\StringTok{ }\NormalTok{acceleration , }\DataTypeTok{data =}\NormalTok{ mpg)}
\KeywordTok{summary}\NormalTok{ (fit_}\DecValTok{5}\NormalTok{)}
\end{Highlighting}
\end{Shaded}

\begin{verbatim}
## 
## Call:
## lm(formula = mpg ~ cylinders + displacement + horsepower + weight + 
##     acceleration, data = mpg)
## 
## Residuals:
##      Min       1Q   Median       3Q      Max 
## -11.5816  -2.8618  -0.3404   2.2438  16.3416 
## 
## Coefficients:
##                Estimate Std. Error t value Pr(>|t|)    
## (Intercept)   4.626e+01  2.669e+00  17.331   <2e-16 ***
## cylinders    -3.979e-01  4.105e-01  -0.969   0.3330    
## displacement -8.313e-05  9.072e-03  -0.009   0.9927    
## horsepower   -4.526e-02  1.666e-02  -2.716   0.0069 ** 
## weight       -5.187e-03  8.167e-04  -6.351    6e-10 ***
## acceleration -2.910e-02  1.258e-01  -0.231   0.8171    
## ---
## Signif. codes:  0 '***' 0.001 '**' 0.01 '*' 0.05 '.' 0.1 ' ' 1
## 
## Residual standard error: 4.247 on 386 degrees of freedom
## Multiple R-squared:  0.7077, Adjusted R-squared:  0.7039 
## F-statistic: 186.9 on 5 and 386 DF,  p-value: < 2.2e-16
\end{verbatim}

\hypertarget{multicollinearity-is-reflected-in-the-fact-that-three-of-the-variables---cylinders-displacement-and-acceleration--all-appears-to-be-insignificant-overall-f-statistic-is-significant-but-3-out-of-5-of-the-t-values-are-nonsignificant.}{%
\subsubsection{Multicollinearity is reflected in the fact that three of
the variables - cylinders, displacement and acceleration- all appears to
be insignificant-- overall F statistic is significant, but 3 out of 5 of
the t values are
nonsignificant.}\label{multicollinearity-is-reflected-in-the-fact-that-three-of-the-variables---cylinders-displacement-and-acceleration--all-appears-to-be-insignificant-overall-f-statistic-is-significant-but-3-out-of-5-of-the-t-values-are-nonsignificant.}}

\hypertarget{c}{%
\subsubsection{(c)}\label{c}}

\begin{Shaded}
\begin{Highlighting}[]
\NormalTok{fit_}\DecValTok{4}\NormalTok{ <-}\StringTok{ }\KeywordTok{lm}\NormalTok{(mpg }\OperatorTok{~}\StringTok{ }\NormalTok{cylinders}\OperatorTok{+}\NormalTok{horsepower}\OperatorTok{+}\NormalTok{weight }\OperatorTok{+}\StringTok{ }\NormalTok{acceleration , }\DataTypeTok{data =}\NormalTok{ mpg)}
\KeywordTok{summary}\NormalTok{ (fit_}\DecValTok{4}\NormalTok{)}
\end{Highlighting}
\end{Shaded}

\begin{verbatim}
## 
## Call:
## lm(formula = mpg ~ cylinders + horsepower + weight + acceleration, 
##     data = mpg)
## 
## Residuals:
##      Min       1Q   Median       3Q      Max 
## -11.5807  -2.8628  -0.3409   2.2427  16.3422 
## 
## Coefficients:
##                Estimate Std. Error t value Pr(>|t|)    
## (Intercept)  46.2739915  2.4481591  18.902  < 2e-16 ***
## cylinders    -0.4004602  0.3032615  -1.321  0.18744    
## horsepower   -0.0452970  0.0160604  -2.820  0.00504 ** 
## weight       -0.0051902  0.0007341  -7.070 7.26e-12 ***
## acceleration -0.0289828  0.1248944  -0.232  0.81661    
## ---
## Signif. codes:  0 '***' 0.001 '**' 0.01 '*' 0.05 '.' 0.1 ' ' 1
## 
## Residual standard error: 4.242 on 387 degrees of freedom
## Multiple R-squared:  0.7077, Adjusted R-squared:  0.7047 
## F-statistic: 234.2 on 4 and 387 DF,  p-value: < 2.2e-16
\end{verbatim}

\hypertarget{it-decreases-the-value-of-cylinders-and-makes-it-more-significant.-and-the-change-increased-ajusted-r-squated-slightly}{%
\subsubsection{It decreases the value of cylinders and makes it more
significant. And the change increased ajusted r squated
slightly}\label{it-decreases-the-value-of-cylinders-and-makes-it-more-significant.-and-the-change-increased-ajusted-r-squated-slightly}}

\hypertarget{d}{%
\subsubsection{(d)}\label{d}}

\begin{Shaded}
\begin{Highlighting}[]
\KeywordTok{plot}\NormalTok{(fit_}\DecValTok{4}\NormalTok{,}\DecValTok{2}\NormalTok{)}
\end{Highlighting}
\end{Shaded}

\includegraphics{MSIA401HW4_files/figure-latex/unnamed-chunk-7-1.pdf}

\begin{Shaded}
\begin{Highlighting}[]
\KeywordTok{plot}\NormalTok{(fit_}\DecValTok{4}\NormalTok{,}\DecValTok{1}\NormalTok{)}
\end{Highlighting}
\end{Shaded}

\includegraphics{MSIA401HW4_files/figure-latex/unnamed-chunk-7-2.pdf}

\begin{Shaded}
\begin{Highlighting}[]
\KeywordTok{vif}\NormalTok{ (fit_}\DecValTok{4}\NormalTok{)}
\end{Highlighting}
\end{Shaded}

\begin{verbatim}
##    cylinders   horsepower       weight acceleration 
##     5.815763     8.305342     8.449468     2.580303
\end{verbatim}

\begin{Shaded}
\begin{Highlighting}[]
\NormalTok{fit_inverse <-}\StringTok{ }\KeywordTok{lm}\NormalTok{((}\DecValTok{100}\OperatorTok{/}\NormalTok{mpg) }\OperatorTok{~}\StringTok{ }\NormalTok{cylinders}\OperatorTok{+}\NormalTok{horsepower}\OperatorTok{+}\NormalTok{weight }\OperatorTok{+}\StringTok{ }\NormalTok{acceleration , }\DataTypeTok{data =}\NormalTok{ mpg)}
\KeywordTok{plot}\NormalTok{(fit_inverse,}\DecValTok{2}\NormalTok{)}
\end{Highlighting}
\end{Shaded}

\includegraphics{MSIA401HW4_files/figure-latex/unnamed-chunk-7-3.pdf}

\begin{Shaded}
\begin{Highlighting}[]
\KeywordTok{plot}\NormalTok{(fit_inverse,}\DecValTok{1}\NormalTok{)}
\end{Highlighting}
\end{Shaded}

\includegraphics{MSIA401HW4_files/figure-latex/unnamed-chunk-7-4.pdf}

\begin{Shaded}
\begin{Highlighting}[]
\KeywordTok{vif}\NormalTok{(fit_inverse)}
\end{Highlighting}
\end{Shaded}

\begin{verbatim}
##    cylinders   horsepower       weight acceleration 
##     5.815763     8.305342     8.449468     2.580303
\end{verbatim}

\hypertarget{yes-the-transformation-removed-the-flaws-of-the-previous-model.-variability-of-the-residuals-is-more-consistent.-vif-does-not-change-and-thats-because-we-are-only-doing-transformation-on-the-ys-and-hence-the-correlation-between-the-xs-still-remains-the-same.}{%
\subsubsection{Yes the transformation removed the flaws of the previous
model. Variability of the residuals is more consistent. VIF does not
change, and that's because we are only doing transformation on the y's
(and hence the correlation between the x's still remains the
same).}\label{yes-the-transformation-removed-the-flaws-of-the-previous-model.-variability-of-the-residuals-is-more-consistent.-vif-does-not-change-and-thats-because-we-are-only-doing-transformation-on-the-ys-and-hence-the-correlation-between-the-xs-still-remains-the-same.}}

\#\#\#(e)

\begin{Shaded}
\begin{Highlighting}[]
\DecValTok{100}\OperatorTok{/}\KeywordTok{predict}\NormalTok{(fit_inverse, }\DataTypeTok{newdata =} \KeywordTok{data.frame}\NormalTok{(}\DataTypeTok{cylinders =} \DecValTok{6}\NormalTok{, }\DataTypeTok{horsepower=} \DecValTok{105}\NormalTok{, }\DataTypeTok{weight =} \DecValTok{3000}\NormalTok{, }\DataTypeTok{acceleration =} \DecValTok{15}\NormalTok{), }\DataTypeTok{interval =} \StringTok{"predict"}\NormalTok{)}
\end{Highlighting}
\end{Shaded}

\begin{verbatim}
##        fit    lwr      upr
## 1 20.59855 28.903 16.00109
\end{verbatim}

\hypertarget{section-2}{%
\subsection{4.12}\label{section-2}}

\#\#\#(a)

\begin{Shaded}
\begin{Highlighting}[]
\KeywordTok{library}\NormalTok{ (tidyverse)}
\end{Highlighting}
\end{Shaded}

\begin{verbatim}
## -- Attaching packages ---------- tidyverse 1.3.0 --
\end{verbatim}

\begin{verbatim}
## v ggplot2 3.3.2     v purrr   0.3.4
## v tibble  3.0.3     v stringr 1.4.0
## v tidyr   1.1.1     v forcats 0.5.0
## v readr   1.3.1
\end{verbatim}

\begin{verbatim}
## -- Conflicts ------------- tidyverse_conflicts() --
## x dplyr::filter() masks stats::filter()
## x dplyr::lag()    masks stats::lag()
## x dplyr::recode() masks car::recode()
## x purrr::some()   masks car::some()
\end{verbatim}

\begin{Shaded}
\begin{Highlighting}[]
\NormalTok{acetylene <-}\StringTok{ }
\StringTok{  }\KeywordTok{tribble}\NormalTok{(}\OperatorTok{~}\NormalTok{x1,  }\OperatorTok{~}\NormalTok{x2,   }\OperatorTok{~}\NormalTok{x3,   }\OperatorTok{~}\NormalTok{y,}
           \DecValTok{1300}\NormalTok{, }\FloatTok{7.5}\NormalTok{,  }\FloatTok{0.0120}\NormalTok{, }\FloatTok{49.0}\NormalTok{,}
           \DecValTok{1300}\NormalTok{, }\DecValTok{9}\NormalTok{,    }\FloatTok{0.0120}\NormalTok{, }\FloatTok{50.2}\NormalTok{,}
           \DecValTok{1300}\NormalTok{, }\DecValTok{11}\NormalTok{,   }\FloatTok{0.0115}\NormalTok{, }\FloatTok{50.5}\NormalTok{,}
           \DecValTok{1300}\NormalTok{, }\FloatTok{13.5}\NormalTok{, }\FloatTok{0.0130}\NormalTok{, }\FloatTok{48.5}\NormalTok{,}
           \DecValTok{1300}\NormalTok{, }\DecValTok{17}\NormalTok{,   }\FloatTok{0.0135}\NormalTok{, }\FloatTok{47.5}\NormalTok{,}
           \DecValTok{1300}\NormalTok{, }\DecValTok{23}\NormalTok{,   }\FloatTok{0.0120}\NormalTok{, }\FloatTok{44.5}\NormalTok{,}
           \DecValTok{1200}\NormalTok{, }\FloatTok{5.3}\NormalTok{,  }\FloatTok{0.0400}\NormalTok{, }\FloatTok{28.0}\NormalTok{,}
           \DecValTok{1200}\NormalTok{, }\FloatTok{7.5}\NormalTok{,  }\FloatTok{0.0380}\NormalTok{, }\FloatTok{31.5}\NormalTok{,}
           \DecValTok{1200}\NormalTok{, }\DecValTok{11}\NormalTok{,   }\FloatTok{0.0320}\NormalTok{, }\FloatTok{34.5}\NormalTok{,}
           \DecValTok{1200}\NormalTok{, }\FloatTok{13.5}\NormalTok{, }\FloatTok{0.0260}\NormalTok{, }\FloatTok{35.0}\NormalTok{,}
           \DecValTok{1200}\NormalTok{, }\DecValTok{17}\NormalTok{,   }\FloatTok{0.0340}\NormalTok{, }\FloatTok{38.0}\NormalTok{,}
           \DecValTok{1200}\NormalTok{, }\DecValTok{23}\NormalTok{,   }\FloatTok{0.0410}\NormalTok{, }\FloatTok{38.5}\NormalTok{,}
           \DecValTok{1100}\NormalTok{, }\FloatTok{5.3}\NormalTok{,  }\FloatTok{0.0840}\NormalTok{, }\FloatTok{15.0}\NormalTok{,}
           \DecValTok{1100}\NormalTok{, }\FloatTok{7.5}\NormalTok{,  }\FloatTok{0.0980}\NormalTok{, }\FloatTok{17.0}\NormalTok{,}
           \DecValTok{1100}\NormalTok{, }\DecValTok{11}\NormalTok{,   }\FloatTok{0.0920}\NormalTok{, }\FloatTok{20.5}\NormalTok{,}
           \DecValTok{1100}\NormalTok{, }\DecValTok{17}\NormalTok{,   }\FloatTok{0.0860}\NormalTok{, }\FloatTok{29.5}
\NormalTok{           )}

\NormalTok{fitx1x2 <-}\StringTok{ }\KeywordTok{lm}\NormalTok{ (x1 }\OperatorTok{~}\StringTok{ }\NormalTok{x2, }\DataTypeTok{data =}\NormalTok{ acetylene)}
\NormalTok{fixx2x3 <-}\StringTok{ }\KeywordTok{lm}\NormalTok{ (x2 }\OperatorTok{~}\NormalTok{x3, }\DataTypeTok{data =}\NormalTok{ acetylene)}
\NormalTok{fixx1x3 <-}\StringTok{ }\KeywordTok{lm}\NormalTok{ (x1}\OperatorTok{~}\NormalTok{x3, }\DataTypeTok{data =}\NormalTok{ acetylene)}
\KeywordTok{plot}\NormalTok{(fitx1x2)}
\end{Highlighting}
\end{Shaded}

\includegraphics{MSIA401HW4_files/figure-latex/unnamed-chunk-9-1.pdf}
\includegraphics{MSIA401HW4_files/figure-latex/unnamed-chunk-9-2.pdf}
\includegraphics{MSIA401HW4_files/figure-latex/unnamed-chunk-9-3.pdf}
\includegraphics{MSIA401HW4_files/figure-latex/unnamed-chunk-9-4.pdf}

\begin{Shaded}
\begin{Highlighting}[]
\KeywordTok{plot}\NormalTok{(acetylene[}\DecValTok{1}\OperatorTok{:}\DecValTok{3}\NormalTok{])}
\end{Highlighting}
\end{Shaded}

\includegraphics{MSIA401HW4_files/figure-latex/unnamed-chunk-9-5.pdf}
\#\#\# Yes x1 and x3 appears to be highly correlated.

\#\#\#(b)

\begin{Shaded}
\begin{Highlighting}[]
\KeywordTok{cor}\NormalTok{(acetylene)}
\end{Highlighting}
\end{Shaded}

\begin{verbatim}
##            x1         x2         x3          y
## x1  1.0000000  0.2236278 -0.9582041  0.9450377
## x2  0.2236278  1.0000000 -0.2402310  0.3700350
## x3 -0.9582041 -0.2402310  1.0000000 -0.9139777
## y   0.9450377  0.3700350 -0.9139777  1.0000000
\end{verbatim}

\begin{Shaded}
\begin{Highlighting}[]
\NormalTok{sumtable <-}\KeywordTok{summary}\NormalTok{(acetylene)}
\NormalTok{sumtable}
\end{Highlighting}
\end{Shaded}

\begin{verbatim}
##        x1             x2              x3                y        
##  Min.   :1100   Min.   : 5.30   Min.   :0.01150   Min.   :15.00  
##  1st Qu.:1175   1st Qu.: 7.50   1st Qu.:0.01275   1st Qu.:29.12  
##  Median :1200   Median :11.00   Median :0.03300   Median :36.50  
##  Mean   :1212   Mean   :12.44   Mean   :0.04031   Mean   :36.11  
##  3rd Qu.:1300   3rd Qu.:17.00   3rd Qu.:0.05175   3rd Qu.:47.75  
##  Max.   :1300   Max.   :23.00   Max.   :0.09800   Max.   :50.50
\end{verbatim}

\begin{Shaded}
\begin{Highlighting}[]
\CommentTok{#12.1 Yes I saw that the correlation between x1 and x3 is around 0.9 which is higher than 0.7,so there is multicollinearity. }
\NormalTok{acetylene}
\end{Highlighting}
\end{Shaded}

\begin{verbatim}
## # A tibble: 16 x 4
##       x1    x2     x3     y
##    <dbl> <dbl>  <dbl> <dbl>
##  1  1300   7.5 0.012   49  
##  2  1300   9   0.012   50.2
##  3  1300  11   0.0115  50.5
##  4  1300  13.5 0.013   48.5
##  5  1300  17   0.0135  47.5
##  6  1300  23   0.012   44.5
##  7  1200   5.3 0.04    28  
##  8  1200   7.5 0.038   31.5
##  9  1200  11   0.032   34.5
## 10  1200  13.5 0.026   35  
## 11  1200  17   0.034   38  
## 12  1200  23   0.041   38.5
## 13  1100   5.3 0.084   15  
## 14  1100   7.5 0.098   17  
## 15  1100  11   0.092   20.5
## 16  1100  17   0.086   29.5
\end{verbatim}

\begin{Shaded}
\begin{Highlighting}[]
\NormalTok{fixmodel_old<-}\StringTok{ }\KeywordTok{lm}\NormalTok{ (y}\OperatorTok{~}\StringTok{ }\NormalTok{x1}\OperatorTok{+}\NormalTok{x2}\OperatorTok{+}\NormalTok{x3}\OperatorTok{+}\NormalTok{x1}\OperatorTok{*}\NormalTok{x2}\OperatorTok{+}\NormalTok{x1}\OperatorTok{*}\NormalTok{x3}\OperatorTok{+}\NormalTok{x2}\OperatorTok{*}\NormalTok{x3}\OperatorTok{+}\KeywordTok{I}\NormalTok{(x1}\OperatorTok{^}\DecValTok{2}\NormalTok{)}\OperatorTok{+}\StringTok{ }\KeywordTok{I}\NormalTok{(x2}\OperatorTok{^}\DecValTok{2}\NormalTok{) }\OperatorTok{+}\StringTok{ }\KeywordTok{I}\NormalTok{(x3}\OperatorTok{^}\DecValTok{2}\NormalTok{),acetylene)}
\KeywordTok{vif}\NormalTok{(fixmodel_old)}
\end{Highlighting}
\end{Shaded}

\begin{verbatim}
##           x1           x2           x3      I(x1^2)      I(x2^2)      I(x3^2) 
## 2.856749e+06 1.095614e+04 2.017163e+06 2.501945e+06 6.573359e+01 1.266710e+04 
##        x1:x2        x1:x3        x2:x3 
## 9.802903e+03 1.428092e+06 2.403594e+02
\end{verbatim}

\hypertarget{all-of-the-vifs-are-extremely-high-signifying-strong-multicollinearity.}{%
\subsubsection{All of the VIFs are extremely high, signifying strong
multicollinearity.}\label{all-of-the-vifs-are-extremely-high-signifying-strong-multicollinearity.}}

\hypertarget{c-1}{%
\subsubsection{(c)}\label{c-1}}

\begin{Shaded}
\begin{Highlighting}[]
\KeywordTok{library}\NormalTok{(}\StringTok{"car"}\NormalTok{)}
\NormalTok{acetylene[}\StringTok{"x1"}\NormalTok{] <-}\StringTok{ }\NormalTok{acetylene[}\StringTok{"x1"}\NormalTok{] }\OperatorTok{-}\StringTok{ }\KeywordTok{mean}\NormalTok{(acetylene}\OperatorTok{$}\NormalTok{x1)}
\NormalTok{acetylene[}\StringTok{"x2"}\NormalTok{]  <-acetylene[}\StringTok{"x2"}\NormalTok{] }\OperatorTok{-}\StringTok{ }\KeywordTok{mean}\NormalTok{(acetylene}\OperatorTok{$}\NormalTok{x2)}
\NormalTok{acetylene[}\StringTok{"x3"}\NormalTok{] <-}\StringTok{ }\NormalTok{acetylene[}\StringTok{"x3"}\NormalTok{] }\OperatorTok{-}\StringTok{ }\KeywordTok{mean}\NormalTok{(acetylene}\OperatorTok{$}\NormalTok{x3)}
\NormalTok{fixmodel_new <-}\StringTok{ }\KeywordTok{lm}\NormalTok{ (y}\OperatorTok{~}\StringTok{ }\NormalTok{x1}\OperatorTok{+}\NormalTok{x2}\OperatorTok{+}\NormalTok{x3}\OperatorTok{+}\NormalTok{x1}\OperatorTok{*}\NormalTok{x2}\OperatorTok{+}\NormalTok{x1}\OperatorTok{*}\NormalTok{x3}\OperatorTok{+}\NormalTok{x2}\OperatorTok{*}\NormalTok{x3}\OperatorTok{+}\KeywordTok{I}\NormalTok{(x1}\OperatorTok{^}\DecValTok{2}\NormalTok{)}\OperatorTok{+}\StringTok{ }\KeywordTok{I}\NormalTok{(x2}\OperatorTok{^}\DecValTok{2}\NormalTok{) }\OperatorTok{+}\StringTok{ }\KeywordTok{I}\NormalTok{(x3}\OperatorTok{^}\DecValTok{2}\NormalTok{), acetylene )}
\KeywordTok{vif}\NormalTok{(fixmodel_new)}
\end{Highlighting}
\end{Shaded}

\begin{verbatim}
##          x1          x2          x3     I(x1^2)     I(x2^2)     I(x3^2) 
##  375.247759    1.740631  680.280039 1762.575365    3.164318 1156.766284 
##       x1:x2       x1:x3       x2:x3 
##   31.037059 6563.345193   35.611286
\end{verbatim}

\hypertarget{yes-the-transformation-has-decreased-the-vif-value-but-the-vifs-are-still-very-high.}{%
\subsubsection{Yes the transformation has decreased the vif value, but
the VIFs are still very
high.}\label{yes-the-transformation-has-decreased-the-vif-value-but-the-vifs-are-still-very-high.}}

\hypertarget{gas-mileages-of-cars-ridge-and-lasso-regressions-perform-ridge-and-lasso-regressions-on-the-gas-mileage-data-considered-in-exercise-4.11-and-compare-the-results-with-those-of-ls-regression.}{%
\subsection{5.5 (Gas mileages of cars: Ridge and lasso regressions)
Perform ridge and lasso regressions on the gas mileage data considered
in Exercise 4.11 and compare the results with those of LS
regression.}\label{gas-mileages-of-cars-ridge-and-lasso-regressions-perform-ridge-and-lasso-regressions-on-the-gas-mileage-data-considered-in-exercise-4.11-and-compare-the-results-with-those-of-ls-regression.}}

\begin{Shaded}
\begin{Highlighting}[]
\KeywordTok{library}\NormalTok{(glmnet)}
\end{Highlighting}
\end{Shaded}

\begin{verbatim}
## Loading required package: Matrix
\end{verbatim}

\begin{verbatim}
## 
## Attaching package: 'Matrix'
\end{verbatim}

\begin{verbatim}
## The following objects are masked from 'package:tidyr':
## 
##     expand, pack, unpack
\end{verbatim}

\begin{verbatim}
## Loaded glmnet 4.0-2
\end{verbatim}

\begin{Shaded}
\begin{Highlighting}[]
\NormalTok{mpg <-}\StringTok{ }\KeywordTok{read.csv}\NormalTok{(here}\OperatorTok{::}\KeywordTok{here}\NormalTok{(}\StringTok{"mpg.csv"}\NormalTok{))}
\KeywordTok{head}\NormalTok{(mpg)}
\end{Highlighting}
\end{Shaded}

\begin{verbatim}
##   mpg cylinders displacement horsepower weight acceleration
## 1  18         8          307        130   3504         12.0
## 2  15         8          350        165   3693         11.5
## 3  18         8          318        150   3436         11.0
## 4  16         8          304        150   3433         12.0
## 5  17         8          302        140   3449         10.5
## 6  15         8          429        198   4341         10.0
\end{verbatim}

\begin{Shaded}
\begin{Highlighting}[]
\CommentTok{# ridge regression}
\KeywordTok{set.seed}\NormalTok{(}\DecValTok{12345}\NormalTok{)}
\NormalTok{y <-}\StringTok{ }\NormalTok{mpg}\OperatorTok{$}\NormalTok{mpg}
\NormalTok{x <-}\StringTok{ }\KeywordTok{model.matrix}\NormalTok{(}\KeywordTok{lm}\NormalTok{(mpg}\OperatorTok{~}\NormalTok{., mpg)) }
\NormalTok{ridgefit <-}\StringTok{ }\KeywordTok{cv.glmnet}\NormalTok{(x, y, }\DataTypeTok{alpha =} \DecValTok{0}\NormalTok{, }\DataTypeTok{lambda =} \KeywordTok{seq}\NormalTok{(}\DecValTok{0}\NormalTok{, }\DecValTok{5}\NormalTok{, }\FloatTok{0.001}\NormalTok{), }\DataTypeTok{nfold =} \DecValTok{10}\NormalTok{)}
\NormalTok{lambdaridge <-}\StringTok{ }\NormalTok{ridgefit}\OperatorTok{$}\NormalTok{lambda.min}
\NormalTok{lambdaridge}
\end{Highlighting}
\end{Shaded}

\begin{verbatim}
## [1] 0.211
\end{verbatim}

\begin{Shaded}
\begin{Highlighting}[]
\NormalTok{small.lambda.index <-}\StringTok{ }\KeywordTok{which}\NormalTok{(ridgefit}\OperatorTok{$}\NormalTok{lambda }\OperatorTok{==}\StringTok{ }\NormalTok{ridgefit}\OperatorTok{$}\NormalTok{lambda.min)}
\NormalTok{small.lambda.betas <-}\StringTok{ }\KeywordTok{coef}\NormalTok{(ridgefit}\OperatorTok{$}\NormalTok{glmnet.fit)[,small.lambda.index]}
\NormalTok{small.lambda.betas}
\end{Highlighting}
\end{Shaded}

\begin{verbatim}
##  (Intercept)  (Intercept)    cylinders displacement   horsepower       weight 
## 45.555448014  0.000000000 -0.418007354 -0.006214789 -0.045755247 -0.004347006 
## acceleration 
## -0.057291887
\end{verbatim}

\begin{Shaded}
\begin{Highlighting}[]
\NormalTok{r2 <-}\StringTok{ }\KeywordTok{max}\NormalTok{(}\DecValTok{1} \OperatorTok{-}\StringTok{ }\NormalTok{ridgefit}\OperatorTok{$}\NormalTok{cvm}\OperatorTok{/}\KeywordTok{var}\NormalTok{(y))}
\NormalTok{r2}
\end{Highlighting}
\end{Shaded}

\begin{verbatim}
## [1] 0.7006621
\end{verbatim}

\begin{Shaded}
\begin{Highlighting}[]
\CommentTok{# the fit is: y = -0.42x1 - 0.006x2  - 0.046x3 - 0.004x4 - 0.057x5 + 45.56. R squared is 0.7006621.}

\CommentTok{# compare with coefficients of the ls fit}
\KeywordTok{coef}\NormalTok{(fit_}\DecValTok{5}\NormalTok{)}
\end{Highlighting}
\end{Shaded}

\begin{verbatim}
##   (Intercept)     cylinders  displacement    horsepower        weight 
##  4.626431e+01 -3.979284e-01 -8.313012e-05 -4.525708e-02 -5.186917e-03 
##  acceleration 
## -2.910471e-02
\end{verbatim}

\begin{Shaded}
\begin{Highlighting}[]
\CommentTok{# the coefficients in the ridge regression model differ slightly from those in the original ls model. The coefficients within the ridge regression model have become more neutralized.}

\CommentTok{# lasso regression}
\NormalTok{lassofit <-}\StringTok{ }\KeywordTok{cv.glmnet}\NormalTok{(x, y, }\DataTypeTok{alpha =} \DecValTok{1}\NormalTok{, }\DataTypeTok{lambda =} \KeywordTok{seq}\NormalTok{(}\DecValTok{0}\NormalTok{, }\DecValTok{5}\NormalTok{, }\FloatTok{0.001}\NormalTok{), }\DataTypeTok{nfold =} \DecValTok{10}\NormalTok{)}
\NormalTok{lambdalasso <-}\StringTok{ }\NormalTok{lassofit}\OperatorTok{$}\NormalTok{lambda.min}
\NormalTok{lambdalasso}
\end{Highlighting}
\end{Shaded}

\begin{verbatim}
## [1] 0.107
\end{verbatim}

\begin{Shaded}
\begin{Highlighting}[]
\NormalTok{small.lambda.index <-}\StringTok{ }\KeywordTok{which}\NormalTok{(lassofit}\OperatorTok{$}\NormalTok{lambda }\OperatorTok{==}\StringTok{ }\NormalTok{lassofit}\OperatorTok{$}\NormalTok{lambda.min)}
\NormalTok{small.lambda.betas <-}\StringTok{ }\KeywordTok{coef}\NormalTok{(lassofit}\OperatorTok{$}\NormalTok{glmnet.fit)[,small.lambda.index]}
\NormalTok{small.lambda.betas}
\end{Highlighting}
\end{Shaded}

\begin{verbatim}
##   (Intercept)   (Intercept)     cylinders  displacement    horsepower 
## 45.3433472768  0.0000000000 -0.3593427798 -0.0002114289 -0.0413759469 
##        weight  acceleration 
## -0.0052282327  0.0000000000
\end{verbatim}

\begin{Shaded}
\begin{Highlighting}[]
\NormalTok{r2 <-}\StringTok{ }\KeywordTok{max}\NormalTok{(}\DecValTok{1} \OperatorTok{-}\StringTok{ }\NormalTok{lassofit}\OperatorTok{$}\NormalTok{cvm}\OperatorTok{/}\KeywordTok{var}\NormalTok{(y))}
\NormalTok{r2}
\end{Highlighting}
\end{Shaded}

\begin{verbatim}
## [1] 0.6985505
\end{verbatim}

\begin{Shaded}
\begin{Highlighting}[]
\CommentTok{# the fit is: y = -0.36x1 - 0.0002x2 - 0.041x3 - 0.005x4 + 45.34. R squared is 0.6985505.}
\CommentTok{# the lasso regression dropped the acceleration term entirely and set the effect of displacement and weight to minimum.}
\end{Highlighting}
\end{Shaded}

\hypertarget{section-3}{%
\subsection{5.6}\label{section-3}}

\begin{Shaded}
\begin{Highlighting}[]
\CommentTok{#acetylene = read.csv("acetylene.csv")}
\NormalTok{y =}\StringTok{ }\NormalTok{acetylene}\OperatorTok{$}\NormalTok{y}
\NormalTok{x =}\StringTok{ }\KeywordTok{model.matrix}\NormalTok{(y }\OperatorTok{~}\StringTok{ }\NormalTok{., acetylene)}

\CommentTok{# Ridge Regression}
\KeywordTok{set.seed}\NormalTok{(}\DecValTok{42}\NormalTok{)}
\NormalTok{ridge_cv =}\StringTok{  }\KeywordTok{cv.glmnet}\NormalTok{(x, y, }\DataTypeTok{alpha =} \DecValTok{0}\NormalTok{, }\DataTypeTok{lambda =} \KeywordTok{seq}\NormalTok{(}\DecValTok{0}\NormalTok{, }\DecValTok{5}\NormalTok{, }\FloatTok{0.001}\NormalTok{), }\DataTypeTok{nfold =} \DecValTok{4}\NormalTok{)}
\NormalTok{lambda_ridge =}\StringTok{ }\NormalTok{ridge_cv}\OperatorTok{$}\NormalTok{lambda.min}
\NormalTok{ridge_coef =}\StringTok{ }\KeywordTok{coef}\NormalTok{(ridge_cv}\OperatorTok{$}\NormalTok{glmnet.fit)[,}\KeywordTok{which}\NormalTok{(ridge_cv}\OperatorTok{$}\NormalTok{lambda }\OperatorTok{==}\StringTok{ }\NormalTok{lambda_ridge)]}

\CommentTok{# Lasso Regression}
\KeywordTok{set.seed}\NormalTok{(}\DecValTok{43}\NormalTok{)}
\NormalTok{lasso_cv =}\StringTok{ }\KeywordTok{cv.glmnet}\NormalTok{(x, y, }\DataTypeTok{alpha =} \DecValTok{1}\NormalTok{, }\DataTypeTok{lambda =} \KeywordTok{seq}\NormalTok{(}\DecValTok{0}\NormalTok{, }\DecValTok{5}\NormalTok{, }\FloatTok{0.001}\NormalTok{), }\DataTypeTok{nfold =} \DecValTok{4}\NormalTok{)}
\NormalTok{lambda_lasso =}\StringTok{ }\NormalTok{lasso_cv}\OperatorTok{$}\NormalTok{lambda.min}
\NormalTok{lasso_coef =}\StringTok{ }\KeywordTok{coef}\NormalTok{(lasso_cv}\OperatorTok{$}\NormalTok{glmnet.fit)[,}\KeywordTok{which}\NormalTok{(lasso_cv}\OperatorTok{$}\NormalTok{lambda }\OperatorTok{==}\StringTok{ }\NormalTok{lambda_lasso)]}

\CommentTok{# LS Regression}
\NormalTok{ls_fit =}\StringTok{ }\KeywordTok{lm}\NormalTok{(y }\OperatorTok{~}\StringTok{ }\NormalTok{., acetylene)}
\NormalTok{ls_coef =}\StringTok{ }\NormalTok{ls_fit}\OperatorTok{$}\NormalTok{coefficients}
\end{Highlighting}
\end{Shaded}

\begin{Shaded}
\begin{Highlighting}[]
\CommentTok{# Print estimated coefficients without scientific notation}
\CommentTok{# Ridge coefficients}
\KeywordTok{format}\NormalTok{(ridge_coef, }\DataTypeTok{scientific =} \OtherTok{FALSE}\NormalTok{)}
\end{Highlighting}
\end{Shaded}

\begin{verbatim}
##   (Intercept)   (Intercept)            x1            x2            x3 
## " 36.1062500" "  0.0000000" "  0.1064389" "  0.3406627" "-67.8145213"
\end{verbatim}

\begin{Shaded}
\begin{Highlighting}[]
\CommentTok{# Lasso coefficients}
\KeywordTok{format}\NormalTok{(lasso_coef, }\DataTypeTok{scientific =} \OtherTok{FALSE}\NormalTok{)}
\end{Highlighting}
\end{Shaded}

\begin{verbatim}
##  (Intercept)  (Intercept)           x1           x2           x3 
## "36.1062500" " 0.0000000" " 0.1188331" " 0.1867655" "-9.8979066"
\end{verbatim}

\begin{Shaded}
\begin{Highlighting}[]
\CommentTok{# LS coefficients}
\KeywordTok{format}\NormalTok{(ls_coef, }\DataTypeTok{scientific =} \OtherTok{FALSE}\NormalTok{)}
\end{Highlighting}
\end{Shaded}

\begin{verbatim}
##   (Intercept)            x1            x2            x3 
## " 36.1062500" "  0.1268539" "  0.3481576" "-19.0216969"
\end{verbatim}

\hypertarget{for-the-ridge-regression-all-estimated-coefficients-are-smaller-than-their-corresponding-ls-estimations-in-absolute-values.-in-other-words-the-ridge-regression-shrinks-all-coefficients-smoothly.}{%
\subsection{For the ridge regression, all estimated coefficients are
smaller than their corresponding LS estimations in absolute values. In
other words, the ridge regression shrinks all coefficients
smoothly.}\label{for-the-ridge-regression-all-estimated-coefficients-are-smaller-than-their-corresponding-ls-estimations-in-absolute-values.-in-other-words-the-ridge-regression-shrinks-all-coefficients-smoothly.}}

\hypertarget{for-the-lasso-regression-variables-x1-x3-and-x1x2-have-estimated-coefficients-of-0-meaning-that-they-are-dropped-from-the-model.}{%
\subsection{For the lasso regression, variables x1, x3, and x1x2 have
estimated coefficients of 0, meaning that they are dropped from the
model.}\label{for-the-lasso-regression-variables-x1-x3-and-x1x2-have-estimated-coefficients-of-0-meaning-that-they-are-dropped-from-the-model.}}

\end{document}
